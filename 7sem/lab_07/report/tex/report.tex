\documentclass{bmstu}

\begin{document}

\makereporttitle
{Информатика и системы управления (ИУ)}
{Программное обеспечение ЭВМ и информационные технологии (ИУ7)}
{\textbf{7}}
{Моделирование}
{Моделирование работы системы массового обслуживания (GPSS)}
{}
{ИУ7-73Б}
{К.Э. Ковалец}
{И.В. Рудаков}


\setcounter{page}{2}
\renewcommand{\contentsname}{Содержание} 
\tableofcontents

\chapter{Задание}

Для выполнения лабораторной необходимо смоделировать работу системы массового обслуживания, состоящую из генератора и обслуживающего аппарата. Генератор работает по равномерному закону распределения, а обслуживающий аппарат --- по закону распределения Эрланга (в соответствии с вариантом из лабораторной работы №1). Необходимо определить максимальную длину очереди без потерь. Предусмотреть возможность возврата обработанной заявки обратно на вход обслуживающего аппарата (задается вероятностью). Реализовать на языке имитационного моделирования GPSS.

\chapter{Теоретическая часть}

\section{Равномерное распределение}

Функция равномерного распределения:

\begin{equation}
    F(x) =
    \begin{cases}
            0, x < a, \\
            \begin{aligned}
                \frac{x -  a}{b - a}, x \in [a, b], 
            \end{aligned}\\
            0, x > b. \\
    \end{cases}
\end{equation}

Функция плотности равномерного распределения:

\begin{equation}
    f(x) =
    \begin{cases}
            \begin{aligned}
                \frac{1}{b - a}, x \in [a, b], 
            \end{aligned}\\
            0, else. \\
    \end{cases}
\end{equation}

\section{Распределение Эрланга}

Функция распределения Эрланга:

\begin{equation}
    \begin{aligned}
        F_k(x) = 1 - e^{-\lambda \cdot x} \cdot \sum_{i = 1}^{k - 1} \frac{(\lambda \cdot x)^i}{i!}.
    \end{aligned}
\end{equation}


Функция плотности распределения Эрланга:

\begin{equation}
    \begin{aligned}
        f_k(x) = \frac{\lambda \cdot (\lambda \cdot x)^{k - 1}}{(k - 1)!} \cdot e^{-\lambda \cdot x}.
    \end{aligned}
\end{equation}

В данных формулах $\lambda$ и $k$ --- положительные параметры распределения $(\lambda \geqslant 0; k = 1, 2, ...)$;
$x \geqslant 0$.


\chapter{Результаты работы}

\section{Листинги программы}

Специальный эрланговский закон можно ввести частным случаем гамма-распределения с помощью функции \texttt{(GAMMA (A,B,C,D))}. В аргументе \texttt{А} записывается номер генератора равномерно распределенных случайных чисел в диапазоне от 0 до 1, который рекомендуется выбирать из диапазона от 1 до 7. Для специального эрланговского закона аргумент \texttt{B} принимается равным 0, в аргумент \texttt{C} записывается среднее значение, а в аргумент \texttt{D} записывается количество фаз.

В листинге \ref{lst:lab_07} представлен код программы.

\mylisting[python]{lab_07.txt}{firstline=1,lastline=13}{Код программы}{lab_07}{}

В листинге \ref{lst:lab_07_report} представлен результат работы программы.

\mylisting[python]{lab_07_report.txt}{firstline=1,lastline=41}{Результат работы программы}{lab_07_report}{}

Из полученного результата видно, что при вероятности возврата 0.1 максимальная длина очереди равна 502.

\end{document}
