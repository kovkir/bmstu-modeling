\documentclass{bmstu}

\begin{document}

\makereporttitle
{Информатика и системы управления (ИУ)}
{Программное обеспечение ЭВМ и информационные технологии (ИУ7)}
{Лабораторной работе №3}
{Моделирование}
{Генераторы псевдослучайных чисел}
{}
{ИУ7-73Б}
{К.Э. Ковалец}
{И.В. Рудаков}


\setcounter{page}{2}
\renewcommand{\contentsname}{Содержание} 
\tableofcontents

\chapter{Задание}

Разработать графический интерфейс, который позволяет сгенерировать последовательность псевдослучайных чисел алгоритмическим и табличным методами, а также рассчитать коэффициенты их случайности. Необходимо предусмотреть возможность ввода чесел.


\chapter{Теоретическая часть}

\section{Генераторы псевдослучайных чисел}

Среди способов получения последовательности псевдослучайных чисел различают:

\begin{itemize}
    \item аппаратаные;
    \item табличные;
    \item алгоритмические.
\end{itemize}

\section{Табличный способ}

Если случайные числа, оформленные в виде таблицы, помещать во внешнюю или оперативную память ЭВМ, предварительно сформировав из них соответствующий файл, то такой способ будет называться табличным. Однако этот способ получения случайных чисел при моделировании систем на ЭВМ обычно рационально использовать при сравнительно небольшом объёме таблицы и, соответственно, файла чисел, когда для хранения можно применять оперативную память. Хранение файла во внешней памяти при частном обращении в процессе статистического моделирования не рационально, так как вызывает увеличение затрат машинного времени при моделировании системы из-за необходимости обращения к внешнему накопителю.

\section{Алгоритмический способ}

Алгоритмический способ --- это способ получения последовательности случайных чисел, основанный на формировании случайных чисел в ЭВМ с использованием специальных алгоритмов и реализующих их программ.

\section{Реализуемый алгоритмический способ}

В качестве используемого метода генерации последовательности случайных чисел был выбран линейный конгруэнтный метод.
Он заключается в том, что каждое последующее число образуется на основе предыдущего по формуле:

\begin{equation}
    \begin{aligned}
        X_{n + 1} = (a \cdot X_{n } + c) \ mod \ m,
    \end{aligned}
\end{equation}

где $a$, $c$ -- коэффициенты, а $m$ -- модуль, которые подобраны специальным образом. В данной лабораторной работе $a$ = 36261, $c$ = 66037, $m$ = 312500.

\section{Критерий случайности}

В качестве критерия случайности было использовано отношение средних арифметических четных и нечетных элементов друг к другу.
Чем ближе коэффициент к 1, тем последовательность более случайна.


\chapter{Результаты работы}

\section{Листинги программы}

В листинге \ref{lst:myRandom} представлен класс $MyRandom$, отвечающий за вычисление случайного числа линейным конгруэнтным методом, а также нахождение коэффициента случайности.

\mylisting[python]{myRandom.py}{firstline=1,lastline=36}{class MyRandom}{myRandom}{}


\clearpage

\section{Демонстрация работы программы}

На рисунках \ref{img:interface} - \ref{img:test} представлены примеры работы программы.

\imgs{interface}{h!}{0.46}{Алгоритмический и табличный методы}

\imgs{test}{h!}{0.7}{Ручной ввод последовательности}

\end{document}
